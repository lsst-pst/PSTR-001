% generated from JIRA project LVV
% using template at /usr/share/miniconda/envs/docsteady-env/lib/python3.13/site-packages/docsteady/templates/tpr.latex.jinja2.
% using docsteady version 0.0.0
% Please do not edit -- update information in Jira instead
\documentclass[PSE,STR,toc]{lsstdoc}
\usepackage{geometry}
\usepackage{longtable,booktabs}
\usepackage{enumitem}
\usepackage{arydshln}
\usepackage{attachfile}
\usepackage{array}
\usepackage{dashrule}
\usepackage{pdfpages}

\newcolumntype{L}[1]{>{\raggedright\let\newline\\\arraybackslash\hspace{0pt}}p{#1}}

\input{meta.tex}

\newcommand{\attachmentsUrl}{https://github.com/\gitorg/\lsstDocType-\lsstDocNum/blob/\gitref/attachments}
\providecommand{\tightlist}{
  \setlength{\itemsep}{0pt}\setlength{\parskip}{0pt}}

\setcounter{tocdepth}{4}

\providecommand{\ul}[1]{\textbf{#1}}

\begin{document}

\def\milestoneName{Survey Strategy Acceptance Test Campaign}
\def\milestoneId{}
\def\product{Survey Strategy}

\setDocCompact{true}

\title{LVV-P105: Survey Strategy Acceptance Test Campaign Test Plan and Report}
\setDocRef{\lsstDocType-\lsstDocNum}
\date{ 2024-10-23 }
\author{ Lynne Jones }

% Most recent last
\setDocChangeRecord{
\addtohist{}{2023-06-05}{First draft}{Lynne Jones }
\addtohist{1.0}{2024-10-23}{Closeout OPSIM-1075}{Lynne Jones}
}

\setDocCurator{Lynne Jones }
\setDocUpstreamLocation{\url{https://github.com/lsst-dm/\lsstDocType-\lsstDocNum}}
\setDocUpstreamVersion{\vcsRevision}



\setDocAbstract{
This is the test plan and report for
\textbf{ Survey Strategy Acceptance Test Campaign},
an LSST milestone pertaining to the Project System Engineering and Commissioning.\\
This document is based on content automatically extracted from the Jira test database on \docDate.
The most recent change to the document repository was on \vcsDate.
}


\maketitle

\section{Introduction}
\label{sect:intro}


\subsection{Objectives}
\label{sect:objectives}

 The primary goal of this acceptance test campaign is to verify those
requirements pertaining to the survey design.~



\subsection{System Overview}
\label{sect:systemoverview}

 This test campaign is intended to verify that the Survey Strategy
satisfies the requirements on the design of survey outlined in the LSST
Science Requirements Document ( \href{https://ls.st/lpm-17}{SRD} ),
ensuring that the survey strategy will deliver the science goals of
LSST.


\subsection{Document Overview}
\label{sect:docoverview}

This document was generated from Jira, obtaining the relevant information from the
\href{https://jira.lsstcorp.org/secure/Tests.jspa\#/testPlan/LVV-P105}{LVV-P105}
~Jira Test Plan and related Test Cycles (
\href{https://jira.lsstcorp.org/secure/Tests.jspa\#/testCycle/LVV-C259}{LVV-C259}
).

Section \ref{sect:intro} provides an overview of the test campaign, the system under test (\product{}),
the applicable documentation, and explains how this document is organized.
Section \ref{sect:testplan} provides additional information about the test plan, like for example the configuration
used for this test or related documentation.
Section \ref{sect:personnel} describes the necessary roles and lists the individuals assigned to them.

Section \ref{sect:overview} provides a summary of the test results, including an overview in Table \ref{table:summary},
an overall assessment statement and suggestions for possible improvements.
Section \ref{sect:detailedtestresults} provides detailed results for each step in each test case.

The current status of test plan \href{https://jira.lsstcorp.org/secure/Tests.jspa\#/testPlan/LVV-P105}{LVV-P105} in Jira is \textbf{ Completed }.

\subsection{References}
\label{sect:references}
\renewcommand{\refname}{}
\bibliography{lsst,refs,books,refs_ads,local}


\newpage
\section{Test Plan Details}
\label{sect:testplan}


\subsection{Data Collection}

  Observing is not required for this test campaign.

\subsection{Verification Environment}
\label{sect:hwconf}
  Verification will be performed using the V3.2 simulations of the survey
cadence.

  \subsection{Entry Criteria}
  Availability of simulations for baseline survey v3.2, availability of
rubin\_sim v1.3 or better.~\\
\strut \\



\subsection{Related Documentation}


No additional documentation provided.


\subsection{PMCS Activity}

Primavera milestones related to the test campaign:
\begin{itemize}
\item None
\end{itemize}


\newpage
\section{Personnel}
\label{sect:personnel}

The personnel involved in the test campaign is shown in the following table.

{\small
\begin{longtable}{p{3cm}p{3cm}p{3cm}p{6cm}}
\hline
\multicolumn{2}{r}{T. Plan \href{https://jira.lsstcorp.org/secure/Tests.jspa\#/testPlan/LVV-P105}{LVV-P105} owner:} &
\multicolumn{2}{l}{\textbf{ Lynne Jones } }\\\hline
\multicolumn{2}{r}{T. Cycle \href{https://jira.lsstcorp.org/secure/Tests.jspa\#/testCycle/LVV-C259}{LVV-C259} owner:} &
\multicolumn{2}{l}{\textbf{
Lynne Jones }
} \\\hline
\textbf{Test Cases} & \textbf{Assigned to} & \textbf{Executed by} & \textbf{Additional Test Personnel} \\ \hline
\href{https://jira.lsstcorp.org/secure/Tests.jspa#/testCase/LVV-T2846}{LVV-T2846}
& {\small Lynne Jones } & {\small Lynne Jones } &
\begin{minipage}[]{6cm}
\smallskip
{\small  }
\medskip
\end{minipage}
\\ \hline
\href{https://jira.lsstcorp.org/secure/Tests.jspa#/testCase/LVV-T2847}{LVV-T2847}
& {\small Lynne Jones } & {\small Lynne Jones } &
\begin{minipage}[]{6cm}
\smallskip
{\small  }
\medskip
\end{minipage}
\\ \hline
\href{https://jira.lsstcorp.org/secure/Tests.jspa#/testCase/LVV-T2848}{LVV-T2848}
& {\small Lynne Jones } & {\small Lynne Jones } &
\begin{minipage}[]{6cm}
\smallskip
{\small  }
\medskip
\end{minipage}
\\ \hline
\href{https://jira.lsstcorp.org/secure/Tests.jspa#/testCase/LVV-T2851}{LVV-T2851}
& {\small Lynne Jones } & {\small Lynne Jones } &
\begin{minipage}[]{6cm}
\smallskip
{\small  }
\medskip
\end{minipage}
\\ \hline
\href{https://jira.lsstcorp.org/secure/Tests.jspa#/testCase/LVV-T2850}{LVV-T2850}
& {\small Lynne Jones } & {\small Lynne Jones } &
\begin{minipage}[]{6cm}
\smallskip
{\small  }
\medskip
\end{minipage}
\\ \hline
\href{https://jira.lsstcorp.org/secure/Tests.jspa#/testCase/LVV-T2849}{LVV-T2849}
& {\small Lynne Jones } & {\small Lynne Jones } &
\begin{minipage}[]{6cm}
\smallskip
{\small  }
\medskip
\end{minipage}
\\ \hline
\end{longtable}
}

\newpage

\section{Test Campaign Overview}
\label{sect:overview}

\subsection{Summary}
\label{sect:summarytable}

{\small
\begin{longtable}{p{2cm}cp{2.3cm}p{8.6cm}p{2.3cm}}
\toprule
\multicolumn{2}{r}{ T. Plan \href{https://jira.lsstcorp.org/secure/Tests.jspa\#/testPlan/LVV-P105}{LVV-P105}:} &
\multicolumn{2}{p{10.9cm}}{\textbf{ Survey Strategy Acceptance Test Campaign }} & Completed \\\hline
\multicolumn{2}{r}{ T. Cycle \href{https://jira.lsstcorp.org/secure/Tests.jspa\#/testCycle/LVV-C259}{LVV-C259}:} &
\multicolumn{2}{p{10.9cm}}{\textbf{ Survey Strategy Acceptance Test Campaign }} & Done \\\hline
\textbf{Test Cases} &  \textbf{Ver.} & \textbf{Status} & \textbf{Comment} & \textbf{Issues} \\\toprule
\href{https://jira.lsstcorp.org/secure/Tests.jspa#/testCase/LVV-T2846}{LVV-T2846}
&  1
\\
 \hfill Execution & LVV-E2927
& Not Executed &
\begin{minipage}[]{9cm}
\smallskip

\medskip
\end{minipage}
&   \\\hline
 \hfill Execution & LVV-E3243
& Pass &
\begin{minipage}[]{9cm}
\smallskip
The requirement here comes from \citeds{LSE-29}, which is intended to be flowed
down from the SRD. The SRD clearly defines minimum requirements, as well
as these design requirements, but the flow-down did not capture the
minimums. We pass the minimum requirements from the SRD.
\medskip
\end{minipage}
&   \\\hline
\href{https://jira.lsstcorp.org/secure/Tests.jspa#/testCase/LVV-T2847}{LVV-T2847}
&  1
\\
 \hfill Execution & LVV-E2928
& Pass &
\begin{minipage}[]{9cm}
\smallskip

\medskip
\end{minipage}
&   \\\hline
 \hfill Execution & LVV-E3244
& Pass &
\begin{minipage}[]{9cm}
\smallskip

\medskip
\end{minipage}
&   \\\hline
\href{https://jira.lsstcorp.org/secure/Tests.jspa#/testCase/LVV-T2848}{LVV-T2848}
&  1
\\
 \hfill Execution & LVV-E2929
& Not Executed &
\begin{minipage}[]{9cm}
\smallskip

\medskip
\end{minipage}
&   \\\hline
 \hfill Execution & LVV-E3245
& Pass &
\begin{minipage}[]{9cm}
\smallskip
The requirement here comes from \citeds{LSE-29}, which is intended to be flowed
down from the SRD. The SRD clearly defines minimum requirements, as well
as these design requirements, but the flow-down did not capture the
minimums. We pass the minimum requirements from the SRD.
\medskip
\end{minipage}
&   \\\hline
\href{https://jira.lsstcorp.org/secure/Tests.jspa#/testCase/LVV-T2851}{LVV-T2851}
&  1
\\
 \hfill Execution & LVV-E2930
& Not Executed &
\begin{minipage}[]{9cm}
\smallskip

\medskip
\end{minipage}
&   \\\hline
 \hfill Execution & LVV-E3247
& Pass &
\begin{minipage}[]{9cm}
\smallskip

\medskip
\end{minipage}
&   \\\hline
\href{https://jira.lsstcorp.org/secure/Tests.jspa#/testCase/LVV-T2850}{LVV-T2850}
&  1
\\
 \hfill Execution & LVV-E2931
& Not Executed &
\begin{minipage}[]{9cm}
\smallskip

\medskip
\end{minipage}
&   \\\hline
 \hfill Execution & LVV-E3249
& Pass &
\begin{minipage}[]{9cm}
\smallskip

\medskip
\end{minipage}
&   \\\hline
\href{https://jira.lsstcorp.org/secure/Tests.jspa#/testCase/LVV-T2849}{LVV-T2849}
&  1
\\
 \hfill Execution & LVV-E2932
& Not Executed &
\begin{minipage}[]{9cm}
\smallskip

\medskip
\end{minipage}
&   \\\hline
 \hfill Execution & LVV-E3246
& Pass &
\begin{minipage}[]{9cm}
\smallskip
The requirement here comes from \citeds{LSE-29}, which is intended to be flowed
down from the SRD. The SRD clearly defines minimum requirements, as well
as these design requirements, but the flow-down did not capture the
minimums. We pass the minimum requirements from the SRD.
\medskip
\end{minipage}
&   \\\hline
\caption{Test Campaign Summary}
\label{table:summary}
\end{longtable}
}

\subsection{Overall Assessment}
\label{sect:overallassessment}

Not yet available.

\subsection{Recommended Improvements}
\label{sect:recommendations}

Not yet available.

\newpage
\section{Detailed Test Results}
\label{sect:detailedtestresults}

\subsection{Test Cycle LVV-C259 }

Open test cycle {\it \href{https://jira.lsstcorp.org/secure/Tests.jspa#/testrun/LVV-C259}{Survey Strategy Acceptance Test Campaign}} in Jira.

Test Cycle name: Survey Strategy Acceptance Test Campaign\\
Status: Done

This test cycle comprises all the test cases for the verification of the
survey strategy

\subsubsection{Software Version/Baseline}
rubin\_sims version 1.3 or newer

\subsubsection{Configuration}
Not provided.

\subsubsection{Test Cases in LVV-C259 Test Cycle}

\paragraph{ LVV-T2846 - Verify survey will cover Asky = 18000 square degrees to a median number
of Nv1Sum = 825 visits per pointing. }\mbox{}\\

Version \textbf{1}.
Status \textbf{Approved}.
Open  \href{https://jira.lsstcorp.org/secure/Tests.jspa#/testCase/LVV-T2846}{\textit{ LVV-T2846 } }
test case in Jira.

Verify that the planned survey strategy will result in sky coverage
meeting Asky area to a median number of Nv1Sum visits.\\
The values of Asky and Nv1Sum used in
\href{https://jira.lsstcorp.org/browse/LVV-308}{LVV-308} are the design
goals for the survey.\\
The median number of visits refers to the median number of visits per
pointing, when calculated across Asky area and is reported via MAF as
fO\_Nv Median.\\
The area on sky are which the minimum (although not median) number of
visits per pointing is Nv1Sum can also be calculated, and is reported
via MAF as fO\_Area.\\
\strut \\

\textbf{ Preconditions}:\\


Execution status: {\bf  }

Final comment:\\



Detailed steps results LVV-C259-LVV-T2846 LVV-E2927-3322:\\
{\bf Note:} Steps "Not Executed" and with No Result are not shown in this report.\\
Detailed steps results LVV-C259-LVV-T2846 LVV-E3243-3638:\\
{\bf Note:} Steps "Not Executed" and with No Result are not shown in this report.\\
\begin{tabular}{p{4cm}p{12cm}}
\toprule
Step LVV-E3243-1 & Step Execution Status: \textbf{ Pass } \\ \hline
\end{tabular}
 Description \\
{\footnotesize
Execute "SkyCoverage" notebook in PSTR-001

}
\hdashrule[0.5ex]{\textwidth}{1pt}{3mm}
  Expected Result \\
{\footnotesize
Asky \textgreater= 18000 sq degrees (15000 sq degrees minimum)\\
Nv1Sum \textgreater= 825 visits per pointing (750 minimum)

}
\hdashrule[0.5ex]{\textwidth}{1pt}{3mm}
  Actual Result \\
{\footnotesize
\begin{verbatim}
Asky is 18454.4. Minimum requirement is 15000, Design requirement is 18000.
Nv1Sum is 805.0. Minimum requirement is 750. Design requirement is 825.
\end{verbatim}

}

\paragraph{ LVV-T2847 - Verify survey will cover RVA1 = 2000 square degrees at timescales
between fastRevisitMin = 40s to fastRevisitMax = 1800 seconds nearly
uniformly. }\mbox{}\\

Version \textbf{1}.
Status \textbf{Approved}.
Open  \href{https://jira.lsstcorp.org/secure/Tests.jspa#/testCase/LVV-T2847}{\textit{ LVV-T2847 } }
test case in Jira.

Verify that the survey strategy will result in coverage of RVA1 at
timescales between fastRevisitMin and fastRevisitMax in a satisfactory
manner.\\
The original statement of "near uniformity" over this time span does not
account for the peak in this timescale caused by standard pairs of
visits; (40s to 1800s=30 minutes; current pairs are acquired at between
20-30 minutes). The intent was to make sure that there was sufficient
coverage at timescales below the pair timing, rather than strictly
providing "uniform" coverage. The metrics in rubin\_sim.maf have been
written to account for the intent of the requirement -\/- that there are
a significant fraction of visits in the timespan 40s - 20 minutes, as
well as visits between 20 - 30 minutes.

\textbf{ Preconditions}:\\


Execution status: {\bf  }

Final comment:\\



Detailed steps results LVV-C259-LVV-T2847 LVV-E2928-3323:\\
{\bf Note:} Steps "Not Executed" and with No Result are not shown in this report.\\
Detailed steps results LVV-C259-LVV-T2847 LVV-E3244-3639:\\
{\bf Note:} Steps "Not Executed" and with No Result are not shown in this report.\\
\begin{tabular}{p{4cm}p{12cm}}
\toprule
Step LVV-E3244-1 & Step Execution Status: \textbf{ Pass } \\ \hline
\end{tabular}
 Description \\
{\footnotesize
Execute SkyCoverage notebook in PSTR-001

}
\hdashrule[0.5ex]{\textwidth}{1pt}{3mm}
  Expected Result \\
{\footnotesize
RVA1 \textgreater= 2000 sq degrees

}
\hdashrule[0.5ex]{\textwidth}{1pt}{3mm}
  Actual Result \\
{\footnotesize
\begin{verbatim}
RVA1 - Area meeting rapid revisit requirement : 29316.7. Design requirement is 2000 sq degrees.
\end{verbatim}

}

\paragraph{ LVV-T2848 - Verify that the survey strategy distributes observations such that the
median proper motion accuracy per coordinate across the main survey area
will be at least SIGpm = 1.0 mas for sources r\textless24. }\mbox{}\\

Version \textbf{1}.
Status \textbf{Approved}.
Open  \href{https://jira.lsstcorp.org/secure/Tests.jspa#/testCase/LVV-T2848}{\textit{ LVV-T2848 } }
test case in Jira.

Verify the survey strategy distributes observations such that the median
proper motion accuracy per coordinate across the main survey area will
be at least SIGpm = 1.0 mas for sources r\textless24.\\
Survey simulations can estimate astrometric accuracy at r=24.0 for each
visit, and then estimate the accuracy of a fit for proper motion
(accounting for the the parallax factor) using the time distribution of
the visits at each point in the main survey.

\textbf{ Preconditions}:\\


Execution status: {\bf  }

Final comment:\\



Detailed steps results LVV-C259-LVV-T2848 LVV-E2929-3324:\\
{\bf Note:} Steps "Not Executed" and with No Result are not shown in this report.\\
Detailed steps results LVV-C259-LVV-T2848 LVV-E3245-3640:\\
{\bf Note:} Steps "Not Executed" and with No Result are not shown in this report.\\
\begin{tabular}{p{4cm}p{12cm}}
\toprule
Step LVV-E3245-1 & Step Execution Status: \textbf{ Pass } \\ \hline
\end{tabular}
 Description \\
{\footnotesize
Execute "Parallax\_ProperMotion" notebook in PSTR-001

}
\hdashrule[0.5ex]{\textwidth}{1pt}{3mm}
  Expected Result \\
{\footnotesize
SIGpm \textless= 1.0 mas (2.0 mas minimum)

}
\hdashrule[0.5ex]{\textwidth}{1pt}{3mm}
  Actual Result \\
{\footnotesize
\begin{verbatim}
Median proper motion uncertainty over the top 18k square degrees (approximately equivalent to the 'WFD') under SRD seeing and depth conditions is at r=24.0 
SIGpm = 1.31 mas.
The design requirement is 1.0 mas, the minimum requirement is 2.0 mas.
\end{verbatim}

}

\paragraph{ LVV-T2851 - Verify the average time between successive visits over the full set of
survey observations, through a survey simulation. }\mbox{}\\

Version \textbf{1}.
Status \textbf{Approved}.
Open  \href{https://jira.lsstcorp.org/secure/Tests.jspa#/testCase/LVV-T2851}{\textit{ LVV-T2851 } }
test case in Jira.

Verify the average expected time between successive visits, as predicted
by survey simulations.\\
\strut \\
Survey simulations use a model of the telescope to estimate slew times
(including filter change times), coupled with scheduler choices for each
successive visit. The times between successive visits can be evaluated
from these simulations.

\textbf{ Preconditions}:\\


Execution status: {\bf  }

Final comment:\\



Detailed steps results LVV-C259-LVV-T2851 LVV-E2930-3325:\\
{\bf Note:} Steps "Not Executed" and with No Result are not shown in this report.\\
Detailed steps results LVV-C259-LVV-T2851 LVV-E3247-3642:\\
{\bf Note:} Steps "Not Executed" and with No Result are not shown in this report.\\
\begin{tabular}{p{4cm}p{12cm}}
\toprule
Step LVV-E3247-1 & Step Execution Status: \textbf{ Pass } \\ \hline
\end{tabular}
 Description \\
{\footnotesize
Execute "Time\_Between\_Visits" notebook in PSTR-001

}
\hdashrule[0.5ex]{\textwidth}{1pt}{3mm}
  Expected Result \\
{\footnotesize
aveVisitInterval \textless{} 10 s

}
\hdashrule[0.5ex]{\textwidth}{1pt}{3mm}
  Actual Result \\
{\footnotesize
\begin{verbatim}
aveVisitInterval: Mean slew time is  7.89 seconds
Requirement is 10 seconds
\end{verbatim}

}

\paragraph{ LVV-T2850 - Verify the median expected time between successive visits, as predicted
by survey simulations. }\mbox{}\\

Version \textbf{1}.
Status \textbf{Approved}.
Open  \href{https://jira.lsstcorp.org/secure/Tests.jspa#/testCase/LVV-T2850}{\textit{ LVV-T2850 } }
test case in Jira.

Verify the median expected time between successive visits, as predicted
by survey simulations.\\
\strut \\
Survey simulations use a model of the telescope to estimate slew times
(including filter change times), coupled with scheduler choices for each
successive visit. The times between successive visits can be evaluated
from these simulations.\\
\strut \\

\textbf{ Preconditions}:\\


Execution status: {\bf  }

Final comment:\\



Detailed steps results LVV-C259-LVV-T2850 LVV-E2931-3326:\\
{\bf Note:} Steps "Not Executed" and with No Result are not shown in this report.\\
Detailed steps results LVV-C259-LVV-T2850 LVV-E3249-3644:\\
{\bf Note:} Steps "Not Executed" and with No Result are not shown in this report.\\
\begin{tabular}{p{4cm}p{12cm}}
\toprule
Step LVV-E3249-1 & Step Execution Status: \textbf{ Pass } \\ \hline
\end{tabular}
 Description \\
{\footnotesize
Execute "Time\_Between\_Visits" notebook in PSTR-001

}
\hdashrule[0.5ex]{\textwidth}{1pt}{3mm}
  Expected Result \\
{\footnotesize
medVisitInterval \textless{} 5 s

}
\hdashrule[0.5ex]{\textwidth}{1pt}{3mm}
  Actual Result \\
{\footnotesize
\begin{verbatim}
medianVisitInterval: Median slew time is  4.81 seconds
Requirement is 5 seconds
\end{verbatim}

}

\paragraph{ LVV-T2849 - Verify the survey strategy distributes observations such that the
parallax uncertainty across the main survey area will be no more than
SIGpar = 3.0 mas, or SIGparRed = 6.0 mas in y band for sources
r\textless24. }\mbox{}\\

Version \textbf{1}.
Status \textbf{Approved}.
Open  \href{https://jira.lsstcorp.org/secure/Tests.jspa#/testCase/LVV-T2849}{\textit{ LVV-T2849 } }
test case in Jira.

Verify the survey strategy distributes observations such that the
parallax uncertainty across the main survey area will be no more than
SIGpar = 3.0 mas, or SIGparRed = 6.0 mas in y band for sources
r\textless24.\\
Survey simulations can estimate astrometric accuracy at r=24.0 for each
visit, and then estimate the uncertainty in resulting parallax fits
using the time distribution of the visits at each point in the main
survey.

\textbf{ Preconditions}:\\


Execution status: {\bf  }

Final comment:\\



Detailed steps results LVV-C259-LVV-T2849 LVV-E2932-3327:\\
{\bf Note:} Steps "Not Executed" and with No Result are not shown in this report.\\
Detailed steps results LVV-C259-LVV-T2849 LVV-E3246-3641:\\
{\bf Note:} Steps "Not Executed" and with No Result are not shown in this report.\\
\begin{tabular}{p{4cm}p{12cm}}
\toprule
Step LVV-E3246-1 & Step Execution Status: \textbf{ Pass } \\ \hline
\end{tabular}
 Description \\
{\footnotesize
Execute "Parallax\_ProperMotion" notebook in PSTR-001

}
\hdashrule[0.5ex]{\textwidth}{1pt}{3mm}
  Expected Result \\
{\footnotesize
SIGpar \textless= 3.0 mas (6.0 mas minimum)\\
SIGparRed \textless= 6.0 mas (10.0 mas minimum)

}
\hdashrule[0.5ex]{\textwidth}{1pt}{3mm}
  Actual Result \\
{\footnotesize
\begin{verbatim}
Median parallax uncertainty over the top 18k square degrees (approximately equivalent to the 'WFD') under SRD seeing and depth conditions at r=24.0 is 
SIGpara = 5.03 mas 
The design requirement is 3.0 mas, the minimum requirement is 6.0 mas.
\end{verbatim}

\begin{verbatim}
Median parallax uncertainty over the top 18k square degrees under SRD seeing and depth conditions, for sources with SNR=10 visible in y band only
SIGparaRed 6.41 mas
The design requirement is 6.0 mas, the minimum requirement is 10.0 mas.
\end{verbatim}

}




% This appendix is put in as part of the template. You may edit and add to it.
% It is not overwritten by Docsteady.

\newpage
\appendix
\section{Documentation}
The verification process is defined in \citeds{LSE-160}.
The use of Docsteady to format Jira information in various test and planing documents is
described in \citeds{DMTN-140} and practical commands are given in \citeds{DMTN-178}.

The process for survey strategy design extends beyond the requirements in the SRD and LSR, 
and responds to the recommendations from the Survey Cadence and Optimization Committee (SCOC). 
More information at \url{https://survey-strategy.lsst.io/scoc/index.html}

The SCOC recommendations on survey strategy included in the baseline survey strategy here (baseline_v3.2) are defined in 
\citeds{PSTN-053} and \citeds{PSTN-055}. 

\section{Acronyms used in this document}\label{sec:acronyms}
\input{acronyms.tex}

\newpage

% Uncomment this if Docsteady makes you additional appendix
%\input{PSTR-001.appendix.tex}

\end{document}
